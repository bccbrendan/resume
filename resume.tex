%%%%%%%%%%%%%%%%%%%%%%%%%%%%%%%%%%%%%%%%%
% Personal Resume
% Author: Brendan Long
%
% Create from template:
% LaTeX Template
% Version 2.0 (8/5/13)
%
% This template has been downloaded from:
% http://www.LaTeXTemplates.com
%
% Original author:
% Trey Hunner (http://www.treyhunner.com/)
%
% Important note:
% This template requires the resume.cls file to be in the same directory as the
% .tex file. The resume.cls file provides the resume style used for structuring the
% document.
%
%%%%%%%%%%%%%%%%%%%%%%%%%%%%%%%%%%%%%%%%%

%----------------------------------------------------------------------------------------
%	PACKAGES AND OTHER DOCUMENT CONFIGURATIONS
%----------------------------------------------------------------------------------------

\documentclass{resume} % Use the custom resume.cls style

\usepackage[left=0.75in,top=0.6in,right=0.75in,bottom=0.6in]{geometry} % Document margins

\name{Brendan Long} % Your name
\address{4424 Gaines Ranch Loop \\ Apt. 1725 \\ Austin, Texas 78735} % address
\address{(512)~$\cdot$~299~$\cdot$~2285 \\ bccbrendan@gmail.com} % Your phone number and email

\begin{document}

%----------------------------------------------------------------------------------------
%	EDUCATION SECTION
%----------------------------------------------------------------------------------------

\begin{rSection}{Education}

{\bf University of Texas, Austin} \hfill {\em May 2012 } \\ 
M.S. in Software Engineering \\
Overall GPA: 3.95

{\bf University of Texas, Austin} \hfill {\em May 2010 } \\ 
B.S. in Electrical and Computer Engineering \\
Member of Eta Kappa Nu \\
Overall GPA: 3.89

\end{rSection}

%----------------------------------------------------------------------------------------
%	WORK EXPERIENCE SECTION
%----------------------------------------------------------------------------------------

\begin{rSection}{Experience}

\begin{rSubsection}{Intel}{June 2012 - Present}{Software Developer}{Austin, TX}
\item Designed and developed C/C++ software library bridging silicon debug software to pre-silicon emulation models. Used to verify new products in every market segment. Received 2 Division Recognition Awards and one Special Recognition Award.
\item Developed several key features of the software connecting debug tools to Intel's Direct Connect Interrface. Received a Q1'2016 DRA.
\item Drove company-wide adoption of 3rd party debug tools by developing compatibility software for existing use cases. Developed and provided training for new tools. Received Q2'13 Transformation Award, Q3'13 DRA, Q4'14 DRA, and Q4'14 "Above and Beyond" SRA
\end{rSubsection}

%------------------------------------------------

\begin{rSubsection}{Intel}{January 2010 - June 2012}{Validation Engineer Intern}{Austin, TX}
\item Developed embedded HTTP server to provide remote debug access to silicon validation platforms. Received Q2/13 Excellence award for silicon power-on support.
\item Developed Linux kernel module to provide PCI access to FPGA platform for hybrid simulation model.
\end{rSubsection}

%------------------------------------------------

\begin{rSubsection}{Schlumberger}{Summer 2009, Summer 2010}{Software Engineering Intern}{Houstin, TX}
\item Enhanced cable tension monitoring/prediction system for oil well devices. Implemented features requested by oilfield engineers and reduced risk of equipment loss.  
\item Created prototype document classification and search system to enable efficient search of unstructured data.
\end{rSubsection}

\end{rSection}

%----------------------------------------------------------------------------------------
%	TECHNICAL STRENGTHS SECTION
%----------------------------------------------------------------------------------------

\begin{rSection}{Technical Strengths}

\begin{tabular}{ @{} >{\bfseries}l @{\hspace{6ex}} l }
Computer Languages & C/C++, Python, Java, C\# \\
Protocols, Libraries, \& APIs & JTAG, XML, JSON, gmock, gtest \\
Tools & Git, Vim, TeamCity 
\end{tabular}

\end{rSection}

%----------------------------------------------------------------------------------------
%	EXAMPLE SECTION
%----------------------------------------------------------------------------------------

%\begin{rSection}{Section Name}

%Section content\ldots

%\end{rSection}

%----------------------------------------------------------------------------------------

\end{document}
